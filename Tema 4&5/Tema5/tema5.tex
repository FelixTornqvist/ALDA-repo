\documentclass[a5paper,10pt,oneside]{article}

\usepackage[swedish]{babel}

\usepackage[T1]{fontenc}

\usepackage[utf8]{inputenc}

\usepackage{graphicx} 
\usepackage{cite}
\usepackage{url}
\usepackage{ifthen}
\usepackage{listings}	


\def \lstlistingname {Kodexempel}
\lstset{language=Java,tabsize=3,numbers=left,frame=L,floatplacement=hbtp}


\usepackage{ifpdf}
\ifpdf
	\usepackage[hidelinks]{hyperref}
\else
	\usepackage{url}
\fi


\title{Tema 5: Prioritetsköer}
\author{Annika Svedin \url{ansv9785} \and Felix Törnqvist \url{fetr0498}}



\begin{document}

\maketitle

\section*{Muntafrågor}
?
\begin{enumerate}
	\item Förklara hur en prioritetskö fungerar. För Väl Godkänt: Hejsvejs!
	\item Vilka fördelar har en heap över ett träd? I vilka fall skulle man vilja använda en heap över ett träd eller någon annan datastruktur? För Väl Godkänt: förklara hur man lägger in ett värde i en heap samt hur man tar bort ett värde.
	
	\item
	Vilka fördelar och nackdelar har en d-heap över en binär heap? För Väl Godkänt: förklara vilket ordo det blir beroende på hur stort d är.
	\item
	Nämn några olika sätt som man kan implementera en prioritetskö på. För Väl Godkänt: Vad har de för för- och nackdelar?
\end{enumerate}



\end{document}
I