% The format (A5) is selected to facilitate reading on small
% devices and should NOT be changed. 
\documentclass[a5paper,10pt,oneside]{article}
\usepackage[swedish]{babel}

\usepackage[T1]{fontenc}

\usepackage[utf8]{inputenc}

\usepackage{graphicx} 
\usepackage{cite}
\usepackage{url}
\usepackage{ifthen}
\usepackage{listings}	

\def \lstlistingname {Kodexempel}
\lstset{language=Java,tabsize=3,numbers=left,frame=L,floatplacement=hbtp}


\usepackage{ifpdf}
\ifpdf
	\usepackage[hidelinks]{hyperref}
\else
	\usepackage{url}
\fi


\title{Tema 7: Grafer}


\author{Annika Svedin \url{ansv9785} \and Felix Törnqvist \url{fetr0498}}




\begin{document}

\maketitle

\section*{Muntafrågor}

\begin{enumerate}
	\item Beskriv vad en riktad, oriktad repsektive viktad graf är. För väl godkänt: hur kan man kombinera dessa och vilka tillämpningsområden har de.
	
	\item Berätta om olika sätt som det går att implementera en graf. För väl godkänt: berätta också om i vilka fall som implementationerna borde användas, vad är varje implementation bäst på?
	
	
	\item Nämn två algoritmer som tar fram ett minimalt spännande träd. För väl godkänt: beskriv hur de fungerar.
	
	\item Vad skiljer Prims respektive Kruskals algoritmer åt? För väl godkänt: Gör 50 armhävningar.
\end{enumerate}



\end{document}
