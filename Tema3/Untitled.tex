\documentclass[a5paper,10pt,oneside]{article}
\usepackage[swedish]{babel}
\usepackage[T1]{fontenc}
\usepackage[utf8]{inputenc} 


\usepackage{graphicx} 
\usepackage{cite}
\usepackage{url}
\usepackage{ifthen}
\usepackage{listings} 

\def \lstlistingname {Kodexempel}
\lstset{language=Java,tabsize=3,numbers=left,frame=L,floatplacement=hbtp}


\usepackage{ifpdf}
\ifpdf
  \usepackage[hidelinks]{hyperref}
\else
  \usepackage{url}
\fi


\title{Tema NR 3: Träd}
\author{Annika Svedin \url{ansv9785} \and Felix Törnqvist \url{fetr0498}}

\begin{document}


\maketitle

\section*{Muntafrågor}

\begin{enumerate}

\item

Noderna i ett träd har en höjd som går att mäta. Vad består höjden av? I vilka sammanhang är det användbart att mäta höjden?
Frågan ska formuleras om,

\end{enumerate}

\end{document}