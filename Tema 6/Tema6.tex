\documentclass[a5paper,10pt,oneside]{article}

\usepackage[swedish]{babel}
\usepackage[T1]{fontenc}
\usepackage[utf8]{inputenc}

\usepackage{graphicx} 
\usepackage{cite}
\usepackage{url}
\usepackage{ifthen}
\usepackage{listings}	

\def \lstlistingname {Kodexempel}
\lstset{language=Java,tabsize=3,numbers=left,frame=L,floatplacement=hbtp}


\usepackage{ifpdf}
\ifpdf
	\usepackage[hidelinks]{hyperref}
\else
	\usepackage{url}
\fi


\title{Tema 6: Sortering}
\author{Annika Svedin \url{ansv9785} \and Felix Törnqvist \url{fetr0498}}


\begin{document}

\maketitle

\section*{Muntafrågor}

\begin{enumerate}

	\item Du sätter in ett element på en slumpad plats i en redan sorterad lista. Vilken sorteringsalgoritm borde gå snabbast?
		För väl godkänt: Spelar det någon roll var i listan elementet sätts in för den valda algoritmen?	
	
	\item Argumentera för vilket element man helst ska ta som pivot-värde i Quicksort? För Väl Godkänt: förklara de sämsta pivot-värden man kan ta och varför de är dåliga.
	
	\item 
\end{enumerate}



\end{document}
