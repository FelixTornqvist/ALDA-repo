\documentclass[a5paper,10pt,oneside]{article}
\usepackage[swedish]{babel}

\usepackage[T1]{fontenc}

\usepackage[utf8]{inputenc} 

\usepackage{graphicx} 
\usepackage{cite}
\usepackage{url}
\usepackage{ifthen}
\usepackage{listings} 

\def \lstlistingname {Kodexempel}
\lstset{language=Java,tabsize=3,numbers=left,frame=L,floatplacement=hbtp}


\usepackage{ifpdf}
\ifpdf
  \usepackage[hidelinks]{hyperref}
\else
  \usepackage{url}
\fi


\title{Tema NR 1: Linjära datastrukturer}
\author{Annika Svedin \url{ansv9785} \and Felix Törnqvist \url{fetr0498}}

\begin{document}


\maketitle

\section*{Muntafrågor}

\begin{enumerate}

\item Stackimplementaion.

En stack skapas enkelt med hjälp av en ArrayList eller LinkedList, men det finns fall då det är mer fördelaktigt att implementera en egen stack. I vilka fall är det?
För väl godkänt förklara varför det blir så,

\item Sentinel nodes.

En vanlig implementation av en linked list är att den innehåller sk. sentinel nodes. Vad de används de till? 
För väl godkänt beskriv vad de har för fördelar och nackdelar.

\item Ordo.

Vad är the running time, Ordo(?) för att sätta in ett element på första index i en arraylist resp. linked list? 
För väl godkänt förklara varför blir det så.

\item Kö vs. stack.

Vad är skillnaden på en kö och en stack? För väl godkänt, redogör för olika sammanhang de kan vara användbara i.

\item Implementation av köer och stackar.

Vilka är fördelarna och nackdelarna med att implementera köer och stackar med länkad lista jämfört med en array?
För väl godkänt: förklara skillnaderna mellan en kö och en stack, och i vilka sammanhang är de användbara i.


\end{enumerate}

\end{document}
