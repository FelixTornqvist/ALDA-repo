\documentclass[a5paper,10pt,oneside]{article}

\usepackage[swedish]{babel}

\usepackage[T1]{fontenc}

\usepackage[utf8]{inputenc}


\usepackage{graphicx} 
\usepackage{cite}
\usepackage{url}
\usepackage{ifthen}
\usepackage{listings}	

% These two lines set up options for the listings package and
% can be removed if you don't use it, or changed if you, e.g, 
% use another language than Java. 
% For more information about the listings package see:
% ftp://ftp.tex.ac.uk/tex-archive/macros/latex/contrib/listings/listings.pdf
\def \lstlistingname {Kodexempel}
\lstset{language=Java,tabsize=3,numbers=left,frame=L,floatplacement=hbtp}


\usepackage{ifpdf}
\ifpdf
	\usepackage[hidelinks]{hyperref}
\else
	\usepackage{url}
\fi


% Change NR and TITLE below to appropriate values
\title{Tema 2: temporary file}

% Write the name and user namn for all participants in the group here.
% Separate persons with \and
\author{Henrik Bergstrom \url{hebe1234} \and Beatrice Akerblom \url{beak5678}}




\begin{document}

\maketitle

% Here the actual report starts. Everything from here to the start of the
% bibilography should, of course, be removed before you start writing your 
% own text.

\section{This is a heading}

Some dummy text. Some more dummy text, and even more. 

\subsection{Subsection heading}

Here is some type set Java code. 

\begin{lstlisting}
	public void testEmpty() {
		list = new MyALDAList<String>();
		assertEquals(0, list.size());
		assertEquals("[]", list.toString());
	}
\end{lstlisting}

If you don't want to copy the code into the document you can pull it directly from the source file instead:
(uncommented).
%\lstinputlisting{Exempel.java}

Or you can pick out just a few selected lines:
(uncommented).
%\lstinputlisting[firstline=3, lastline=3]{Exempel.java}

To quote a source you use \cite{Weiss}, or \cite[sid. 25]{Weiss}. Sources are kept in the file bibtex.bib.

% Any material used should be properly referenced. This includes the course litterature.
% Latex uses a reference management system called bibtex
\bibliographystyle{plain}
\bibliography{bibtex}
\bibdata{bibtex}


\end{document}
