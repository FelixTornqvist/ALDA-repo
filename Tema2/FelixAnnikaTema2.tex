\documentclass[a5paper,10pt,oneside]{article}
\usepackage[swedish]{babel}
\usepackage[T1]{fontenc}
\usepackage[utf8]{inputenc} 


\usepackage{graphicx} 
\usepackage{cite}
\usepackage{url}
\usepackage{ifthen}
\usepackage{listings} 

\def \lstlistingname {Kodexempel}
\lstset{language=Java,tabsize=3,numbers=left,frame=L,floatplacement=hbtp}


\usepackage{ifpdf}
\ifpdf
  \usepackage[hidelinks]{hyperref}
\else
  \usepackage{url}
\fi


\title{Tema NR 2: Algoritmanalys}
\author{Annika Svedin \url{ansv9785} \and Felix Törnqvist \url{fetr0498}}

\begin{document}


\maketitle
\section*{Analysera körtiden för nedanstående kodsnuttar med hjälp av ordo/big-oh.}

\subsection*{Exempel 1:}

\lstinputlisting[firstline=16, lastline=29]{Sort1.java}


\subsection*{Exempel 2:}

\lstinputlisting{Linear1.java}

\subsection*{Exempel 3:}
Obs. 
Följande kod ska föreställa ett dåligt exempel på en stackimplementation, som inte har O(1) utan istället O(N). Potentiell kuggfråga.
\lstinputlisting{BadStack.java}


\subsection*{Exempel 4:}
\lstinputlisting{BetterStack.java}

\end{document}